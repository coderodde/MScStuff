% --- Template for thesis / report with tktltiki2 class ---
% 
% last updated 2013/02/15 for tkltiki2 v1.02

\documentclass[english]{tktltiki2}

% tktltiki2 automatically loads babel, so you can simply
% give the language parameter (e.g. finnish, swedish, english, british) as
% a parameter for the class: \documentclass[finnish]{tktltiki2}.
% The information on title and abstract is generated automatically depending on
% the language, see below if you need to change any of these manually.
% 
% Class options:
% - grading                 -- Print labels for grading information on the front page.
% - disablelastpagecounter  -- Disables the automatic generation of page number information
%                              in the abstract. See also \numberofpagesinformation{} command below.
%
% The class also respects the following options of article class:
%   10pt, 11pt, 12pt, final, draft, oneside, twoside,
%   openright, openany, onecolumn, twocolumn, leqno, fleqn
%
% The default font size is 11pt. The paper size used is A4, other sizes are not supported.
%
% rubber: module pdftex

% --- General packages ---

\usepackage[utf8]{inputenc}
\usepackage[T1]{fontenc}
\usepackage{lmodern}
\usepackage{microtype}
\usepackage{amsfonts,amsmath,amssymb,amsthm,booktabs,color,enumitem,graphicx}
\usepackage[pdftex,hidelinks]{hyperref}
\usepackage[ruled,vlined,linesnumbered]{algorithm2e}
\usepackage{caption}
\usepackage{float}
\usepackage{subcaption}
\usepackage{pdfpages}

% Algorithm2e environment with "Algoritmi"-caption.
%\newenvironment{finalgo}[1][htb]{
%  \renewcommand{\algorithmcfname}{Algoritmi}
%  \begin{algorithm}[#1]
%}{\end{algorithm}}

% To be able to not numbering individual lines:
\let\oldnl\nl% Store \nl in \oldnl
\newcommand{\nonl}{\renewcommand{\nl}{\let\nl\oldnl}}

% Automatically set the PDF metadata fields
\makeatletter
\AtBeginDocument{\hypersetup{pdftitle = {\@title}, pdfauthor = {\@author}}}
\makeatother

% --- Language-related settings ---
%
% these should be modified according to your language

% babelbib for non-english bibliography using bibtex
\usepackage[fixlanguage]{babelbib}
\selectbiblanguage{english}

% add bibliography to the table of contents
\usepackage[nottoc]{tocbibind}
% tocbibind renames the bibliography, use the following to change it back
\settocbibname{References}

% --- Theorem environment definitions ---

\newtheorem{lau}{Lause}
\newtheorem{lem}[lau]{Lemma}
\newtheorem{kor}[lau]{Korollaari}

\theoremstyle{definition}
\newtheorem{maar}[lau]{Määritelmä}
\newtheorem{ong}{Ongelma}
\newtheorem{alg}[lau]{Algoritmi}
\newtheorem{esim}[lau]{Esimerkki}

\theoremstyle{remark}
\newtheorem*{huom}{Huomautus}

\DeclareMathOperator*{\argmin}{arg\, min}

% --- tktltiki2 options ---
%
% The following commands define the information used to generate title and
% abstract pages. The following entries should be always specified:

\title{Title will be here}
\author{Rodion Efremov}
\date{\today}
\level{Master thesis}
\abstract{Abstract goes here}

\keywords{}

% classification according to ACM Computing Classification System (http://www.acm.org/about/class/)
% This is probably mostly relevant for computer scientists
% uncomment the following; contents of \classification will be printed under the abstract with a title
% "ACM Computing Classification System (CCS):"
% \classification{}

% If the automatic page number counting is not working as desired in your case,
% uncomment the following to manually set the number of pages displayed in the abstract page:
%
% \numberofpagesinformation{16 sivua + 10 sivua liitteissä}
%
% If you are not a computer scientist, you will want to uncomment the following by hand and specify
% your department, faculty and subject by hand:
%
% \faculty{Matemaattis-luonnontieteellinen}
% \department{Tietojenkäsittelytieteen laitos}
% \subject{Tietojenkäsittelytiede}
%
% If you are not from the University of Helsinki, then you will most likely want to set these also:
%
% \university{Helsingin Yliopisto}
% \universitylong{HELSINGIN YLIOPISTO --- HELSINGFORS UNIVERSITET --- UNIVERSITY OF HELSINKI} % displayed on the top of the abstract page
% \city{Helsinki}
%


\begin{document}

% --- Front matter ---

\frontmatter      % roman page numbering for front matter

\maketitle        % title page
\makeabstract     % abstract page

\tableofcontents  % table of contents

% --- Main matter ---

\mainmatter       % clear page, start arabic page numbering

% --- References ---
%
% bibtex is used to generate the bibliography. The babplain style
% will generate numeric references (e.g. [1]) appropriate for theoretical
% computer science. If you need alphanumeric references (e.g [Tur90]), use
%
% \bibliographystyle{babalpha-lf}
%
% instead.

\chapter{Chapt}
\section*{Dummy section}

\subsection{Strongly connected components}
Since our methods require the input graph to be strongly connected, we review here briefly how to algorithmically validate that the input graph exhibits the requirement. A directed graph is called \textit{strongly connected} if and only if for every pair of nodes $u, v$ of the graph $u$ is reachable from $v$, and $v$ is reachable from $u$. Let $u \overset{r}{\sim} v$ denote the aforementioned reachability relation. Now, it is easy to see that
\begin{description}
\item[\textsc{(Reflexivity)}]  $u \overset{r}{\sim} u$, for all $u \in V(G)$.
\item[\textsc{(Symmetry)}] if  $u \overset{r}{\sim} v$, then $v \overset{r}{\sim} u$.
\item[\textsc{(Transitivity)}] If $u \overset{r}{\sim} v$ and $v \overset{r}{\sim} v'$, then $u \overset{r}{\sim} v'$.
\end{description}
The above three properties imply that $\overset{r}{\sim}$ is an equivalence relation, and as such, implies that the graph has a unique partition into strongly connected components.

Algorithms for finding all strongly connected components of a graph in linear time are known. We review three of them below.

\subsubsection{Kosaraju's algorithm}

\begin{algorithm}
\If{$v \not \in S$}{
  $S \leftarrow S \cup \{ v \}$ \\
  \For{$(v, w) \in G(A)$}{
     \textsc{KosarajuVisit}$(G, S, L, w)$ \\
  }
  
  $L \leftarrow \langle v \rangle \circ L$ \\
}
\caption{\textsc{KosarajuVisit}$(G, S, L, v)$}
\label{alg:kosaraju_visit}
\end{algorithm}

\begin{algorithm}
\If{$(u \mapsto r) \not \in \mu$}{
  $\mu(u) \leftarrow r$ \\
  \nonl For all parents of $u$ \\
  \For{$(v, u) \in G(A)$}{
    \textsc{KosarajuAssign}$(G \mu, v, r)$ \\
  }
}
\caption{\textsc{KosarajuAssign}$(G, \mu, u, r)$}
\label{alg:kosaraju_assign}
\end{algorithm}

\begin{algorithm}
$S \leftarrow \varnothing$ \\
$L \leftarrow \langle \rangle$ \\
$\mu \leftarrow \varnothing$ \\
\For{$v \in V(G)$}{
  \textsc{KosarajuVisit}$(G, S, L, v)$ \\
}
\nonl Iterate the list $L$ in its natural order \\
\For{$v \in L$}{
  \textsc{KosarajuAssign}$()$ \\
}
\caption{\textsc{KosarajuSCC}$(G)$}
\label{alg:kosaraju}
\end{algorithm}

\bibliographystyle{babplain-lf}
\bibliography{refs}


% --- Appendices ---

% uncomment the following

% \newpage
% \appendix
% 
% \section{Esimerkkiliite}

\end{document}